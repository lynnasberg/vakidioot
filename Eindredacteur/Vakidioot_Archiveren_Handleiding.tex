\documentclass{article}
\usepackage{hyperref}
\usepackage[utf8]{inputenc}

\title{Handleiding Vakidioot archiveren en verzenden}
\author{Lynn Asberg en Jim Vollebregt}
\date{\today}

\begin{document}

\maketitle

\begin{center}
    Lukt er iets niet of heb je vragen? Stuur dan een mailtje naar Lynn Asberg op \href{mailto:lynnasberg@gmail.com}{lynnasberg@gmail.com}.
\end{center}


\section{Archiveren}

Upload vakidioot en cover:

\begin{enumerate}
    \item Ga naar A-Eskwadraat.nl
    \item Ga naar Vereniging $\to$ Vakidioot $\to$ Archief $\to$ jaar (bijv. 2019-2020)
    \item Klik op het potlood (publisher)
    \item Ga naar `Naar bestanden beheren'
    \item Upload gecomprimeerde Vakidioot (https://pdfcompressor.com/ of ilovepdf.com) en voorkant (als voorkant$n$.jpg)
\end{enumerate}

\noindent Voeg nieuwe Vakidioot toe aan het archief:
\begin{enumerate}
    \item Ga naar Vereniging $\to$ Vakidioot $\to$ Archief $\to$ jaar (bijv. 2019-2020)
    \item Klik op het potlood (publisher)
    \item Klik op `broncode'
    \item Scroll helemaal naar beneden en pas in de goede regel de pdf, titel (\texttt{alt="..."}) en de cover (\texttt{src="..."}) aan zodat ze kloppen.
\end{enumerate}

\noindent Voeg de nieuwe Vakidioot toe aan de homepage:
\begin{enumerate}
    \item Ga naar Vereniging $\to$ Vakidioot
    \item Klik op het potlood (publisher)
    \item Klik op `broncode'
    \item Scroll naar beneden tot je \texttt{<h2>De Vakidioot</h2>} ziet staan
    \item Pas in de regel daaronder de titel, pdf en cover aan naar de nieuwste editie
\end{enumerate}

\section{Pagina voor nieuw jaar toevoegen}
\begin{enumerate}
    \item Ga naar A-Eskwadraat.nl
    \item Ga naar Vereniging $\to$ Vakidioot $\to$ Archief $\to$ jaar (bijv. 2019-2020)
    \item Klik op het potlood (publisher)
    \item Klik op `broncode'
    \item Kopieer de gehele broncode
    \item Ga naar Vereniging $\to$ Vakidioot
    \item Klik op het potlood (publisher)
    \item Klik op `naar bestanden beheren'
    \item Scroll naar beneden naar `Nieuwe map aanmaken' en maak de nieuwe map aan (bijvoorbeeld ``2122'' voor 2021-2022).
    \item Scroll naar boven en klik op het mapicoontje van de map die je net hebt aangemaakt
    \item Maak bij `Nieuw bestand aanmaken' index.html aan
    \item Klik op het potloodje bij index.html
    \item Klik op `broncode'
    \item Vervang de broncode door de zojuist gekopieerde broncode
    \item Verander de jaartallen in de code door de nieuwe jaartallen
    \item Ga naar Vereniging $\to$ Vakidioot $\to$ Archief $\to$ jaar (bijv. 2019-2020, ga naar het vorige jaar)
    \item Klik op het potlood (publisher)
    \item Klik op `broncode'
    \item Zorg dat het nieuwe jaar aanklikbaar is, bijvoorbeeld: \texttt{<a href="../2021/index.html">2020-2021 \&raquo;</a></li>}
\end{enumerate}

\section{Vakidioot mailen}
\begin{enumerate}
    \item Maak een tinyurl aan via Service $\to$ TinyUrl
    \item Klik op `nieuw'
    \item Zet bij `TinyUrl' ``vakid1819-4-titel'', met de juiste jaartallen en nummer
    \item Zet bij `OutgoingUrl' de link naar de geuploadde PDF van de Vakidioot
    \item Stel de email die je naar de Vakidioot-ontvangers wil versturen op in je eigen mailprogramma en mail deze naar \emph{mailings@a-eskwadraat.nl}. Zet als titel ``Vakidioot 1819-4 Titel'' met de juiste jaartallen etc.
    \item Ga naar Service $\to$ Mailings en zoek je mail, klik op de titel.
    \item Klik op het potloodje en pas alles aan naar wens.
    \item Zorg dat je `IsDefinitief' hebt aangevinkt, anders wordt de email niet verzonden
    \item Klik op `Wijzig'
    \item Stuur voor de zekerheid een preview van de mail naar jezelf door op `Preview versturen' te klikken, zo kun je checken of alles klopt.
\end{enumerate}

\section{Drukken}
\begin{enumerate}
    \item Mail de secretaris van A-Es en vraag om de adressenlijst voor de Vakidioot en het aantal donateurs.
    \item De secretaris stuurt je een .tex bestand. Maak hier een Excel bestand van met \texttt{VakidiootAdressenExport.exe} (of via \texttt{VakidiootAdressenExport.cpp} als je niet op Windows zit, dan zul je hem wel zelf moeten compileren).
    \item Kijk hoeveel adressen er zijn, tel hier het aantal donateurs bij op en dit is je oplage.
    \item Ga naar WeTransfer, vul bij ontvanger ``iwan@bladnl.nl''
    \item Type een bericht in de beschrijving, noem hierin de oplage
    \item Druk op `Verstuur'!
\end{enumerate}

\end{document}
